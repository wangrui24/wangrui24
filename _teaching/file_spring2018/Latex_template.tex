\documentclass[12pt]{article}
\usepackage[utf8]{inputenc}
\usepackage[english]{babel}

\usepackage{geometry}
 \geometry{
 a4paper,
 total={170mm,250mm},
 left=31mm,
 right=31mm,
 top=34mm,
 bottom=29mm
 }

\usepackage{amsfonts,amsmath}
\usepackage{mathtools}
\usepackage{graphicx}
\usepackage{amssymb}
\usepackage{amsthm}
\usepackage{tikz-cd}
\usepackage{mathrsfs}
\usepackage{subcaption}
\usepackage{multicol}
\usepackage[colorinlistoftodos]{todonotes}
\usepackage{enumitem}
\usepackage{hyperref}
\usepackage{url}
\hypersetup{
   colorlinks=true,
    linkcolor=blue,
    citecolor=red,
    urlcolor=cyan,
}
%\usepackage{yfonts}
\usepackage{mathdots}

\usepackage{fancyhdr}
\pagestyle{fancy}
\fancyhf{}
\rhead{\textit{Your name}}
\lhead{\small Your title}
\cfoot{\thepage}

\setcounter{MaxMatrixCols}{20} % Enable us to create matrices with more than 10 columns


\title{Your title}

\author{\textit{Your name}\footnote{Your institution}}

\date{\today}

\setlength{\footnotesep}{0.5cm}
\setlength{\skip\footins}{1cm}

\newtheorem{thm}{Theorem}[section]
\newtheorem{lem}[thm]{Lemma}
\newtheorem{cor}[thm]{Corollary}
\newtheorem{prop}[thm]{Proposition}
\newtheorem{defn}[thm]{Definition}
\newtheorem{obs}[thm]{Observation}
\theoremstyle{definition}
\newtheorem{eg}[thm]{Example}
\newtheorem{ex}[thm]{Exercise}


\begin{document}
\maketitle

\begin{abstract}
A concise summary of the whole reading report and main findings
\end{abstract}

\section{Introduction}

A brief overview of the subject or research field that you are going to discuss.

\section{Background}

\begin{defn}
A list of vectors $x_1,...,x_k\in \mathbf{C}^n$ is \emph{orthogonal} if $x_i^*x_j=0$ for all $i\neq j, i,j\in\{1,...,k\}$. If, in addition, $x_i^*x_i=1$ for all $i=1,...,k$ (that is, the vectors are \emph{normalized}), then the list is \emph{orthonormal}.
\end{defn}

\begin{eg}[normalization]
\label{Normalization}
If $y_1,...,y_k\in \mathbf{C}^n$ are orthogonal and nonzero, the vectors $x_1,...,x_k$ defined by $x_i=(y_i^*y_i)^{-\frac{1}{2}}y_i, i=1,...,k$ are orthonormal.
\end{eg}

\begin{thm}
Every orthogonal list of vectors in $\mathbf{C}^n$ is linearly independent.
\end{thm}
\begin{proof}
Suppose that $\{y_1,...,y_k\}$ is an orthogonal set. Normalize them as Example \ref{Normalization} did and obtain an orthonormal list of vectors $\{x_1,...,x_k\}$. Assume that $0=\alpha_1x_1+\cdots+\alpha_k x_k$. Then $0=(\alpha_1x_1+\cdots+\alpha_k x_k)^*(\alpha_1x_1+\cdots+\alpha_k x_k)=\sum\limits_{i,j} \bar{\alpha}_i \alpha_j x_i^*x_j= \sum\limits_{i=1}^k |\alpha_i|^2 x_i^*x_i=\sum\limits_{i=1}^k |\alpha_i|^2$ because the vectors $x_i$ are orthonormal. Thus, all $\alpha_i=0$ and hence $\{x_1,...,x_k\}$ is a linearly independent set, which in turn means that $\{y_1,...,y_k\}$ is linearly independent.
\end{proof}

\section{Method and Evaluation}

Main section of the reading report. It contains your solutions to the exercises, your methodology to address the problem, and your experimental results.

\section{Conclusions}

Summarize what you have written in previous sections and discuss your understandings of possible future research directions.


\vspace{1cm}
\begin{thebibliography}{00}

\bibitem{Horn} Roger A. Horn, Charles R. Johnson (2012) {\em Matrix Analysis, Second Edition.} Cambridge University Press.

% \bibitem{PowerIterWiki} Wikipedia: The Free Encyclopedia \textbf{\em Power iteration.} URL: \url{https://en.wikipedia.org/wiki/Power_iteration} Retrieved 12 April, 2018.

\end{thebibliography}

\end{document}
